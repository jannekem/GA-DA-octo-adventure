% This is a model template for the solutions in computational science. You can find a very useful documentation for LaTeX in Finnish at ftp://ftp.funet.fi/pub/TeX/CTAN/info/lshort/finnish/ or in English at ftp://ftp.funet.fi/pub/TeX/CTAN/info/lshort/english/. The section List of mathematical symbols in Chapter 3 is especially useful for the typesetting of mathematical formulas.

% Compile the document to PDF by command 'pdflatex model.tex' in the terminal. The command must be run twice for the references in the text to be correct.

\documentclass[twoside]{article}

\usepackage{lipsum} % Package to generate dummy text throughout this template

\usepackage[sc]{mathpazo} % Use the Palatino font
\usepackage[T1]{fontenc} % Use 8-bit encoding that has 256 glyphs
\linespread{1.05} % Line spacing - Palatino needs more space between lines
\usepackage{microtype} % Slightly tweak font spacing for aesthetics

\usepackage[hmarginratio=1:1,top=32mm,columnsep=20pt]{geometry} % Document margins
\usepackage{multicol} % Used for the two-column layout of the document
\usepackage[hang, small,labelfont=bf,up,textfont=it,up]{caption} % Custom captions under/above floats in tables or figures
\usepackage{booktabs} % Horizontal rules in tables
\usepackage{float} % Required for tables and figures in the multi-column environment - they need to be placed in specific locations with the [H] (e.g. \begin{table}[H])
\usepackage{hyperref} % For hyperlinks in the PDF
\usepackage[english]{babel}
\usepackage{lettrine} % The lettrine is the first enlarged letter at the beginning of the text
\usepackage{paralist} % Used for the compactitem environment which makes bullet points with less space between them

\usepackage{abstract} % Allows abstract customization
\renewcommand{\abstractnamefont}{\normalfont\bfseries} % Set the "Abstract" text to bold
\renewcommand{\abstracttextfont}{\normalfont\small\itshape} % Set the abstract itself to small italic text

\usepackage{titlesec} % Allows customization of titles
\renewcommand\thesection{\Roman{section}} % Roman numerals for the sections
\renewcommand\thesubsection{\Roman{subsection}} % Roman numerals for subsections
\titleformat{\section}[block]{\large\scshape\centering}{\thesection.}{1em}{} % Change the look of the section titles
\titleformat{\subsection}[block]{\large}{\thesubsection.}{1em}{} % Change the look of the section titles

\usepackage{fancyhdr} % Headers and footers
\pagestyle{fancy} % All pages have headers and footers
\fancyhead{} % Blank out the default header
\fancyfoot{} % Blank out the default footer
\fancyhead[C]{AS-0.3200 Automaatio- ja systeemitekniikan projektity\"ot} % Custom header text
\fancyfoot[RO,LE]{\thepage} % Custom footer text

%----------------------------------------------------------------------------------------
%	TITLE SECTION
%----------------------------------------------------------------------------------------

\title{\vspace{-15mm}\fontsize{24pt}{10pt}\selectfont\textbf{Nature-Inspired Computing (NIC) Methods in Wind Generator Design Intermediate Report}} % Article title

\author{
\large
\textsc{Janne Kemppainen \& Eero J\"arviluoma}\\[2mm] % Your name
\normalsize Aalto University School of Electrical Engineering \\ % Your institution
%\normalsize \href{mailto:john@smith.com}{john@smith.com} % Your email address
\vspace{-5mm}
}
\date{}

%----------------------------------------------------------------------------------------

\begin{document}

\maketitle % Insert title

\thispagestyle{fancy} % All pages have headers and footers

%----------------------------------------------------------------------------------------
%	ABSTRACT
%----------------------------------------------------------------------------------------

\begin{abstract}

\noindent This document is the intermediate report for the project Nature-Inspired Computing (NIC) Methods in Wind Generator Design for the course AS-0.3200 Automaatio- ja systeemitekniikan projektity\"ot. NIC methods include various different algorithms for efficiently finding near-optimal solutions for many optimizing problems. Here we summarize what we have achieved thus far and what are the main problems.

\end{abstract}

%----------------------------------------------------------------------------------------
%	ARTICLE CONTENTS
%----------------------------------------------------------------------------------------

\begin{multicols}{2} % Two-column layout throughout the main article text

\section{Overall Thoughts}

\lettrine[nindent=0em,lines=3]{A}{fter} the first meeting with the instructor we had a pretty clear vision of the project work. The work was divided between two different algorithms, namely Genetic Algorithms (GA) and Particle Swarm Optimization (PSO), and the difficulty was tunable to fit the work requirements. Originally we planned to implement the Differential Evolution algorithm but we switched to PSO.

During the first few weeks we were preoccupied with the project work for the course System Dynamics and had no time to concentrate on this project. After finishing the other project there was more time available and we started looking deeper into this project.

Then there appeared some more obstacles as we both had decided to apply for exchange and we had to find information, plan studies, fill application forms etc. In addition to this, Eero had to allocate his time to apply for a summerjob as the application deadlines were closing in. Therefore we had to delay our initial deadlines.

%------------------------------------------------

\section{The Optimization Problem}
Our optimization problem is a black box model of a wind turbine generator developed by VTT. The model has 14 inputs and 6 outputs. Two of the inputs are integer values and the rest are floating point numbers. In addition to this there are 3 constraints that should be met for the design. 

An ideal generator would produce exactly 3 MW power and have a maximised torque density, minimum weight, high efficiency, high power factor and minimised costs. All the goals can't be optimised simultaneously as they require different parameter combinations. The optimal solution is a compromise where improving some value would deteriorate the others.

%------------------------------------------------

\section{Particle Swarm Optimization}

Implementing the Particle Swarm Optimization algorithm was assigned to Janne. The algorithm was first proposed by James Kennedy and Russel Eberhart in 1995 and the aim is to implement their version of the algorithm with only some small modifications.

The algorithm itself is quite simple. The solution space is first initialized with particles set to random locations with zero velocity. Then the fitness function is evaluated for each particle and the best coordinates are stored. Then the velocities are calculated by the following formula: \cite{pso}
\begin{verbatim}
vx[][]=vx[][]+
 2*rand()*(pbestx[][]-presentx[][])+
 2*rand()*(pbestx[][gbest]-presentx[][])
\end{verbatim}

The positions and velocities are updated until a set criterion is fulfilled. The velocity function has a constant 2 for the stochastic part so that the particles would have the possibility to move over the previous best value. This value yields the best results as there is no good way to individually guess which one of the velocity increments should be weighted more. \cite{pso} 

Our optimization problem is a hybrid of discrete and continuous parameters whereas the original algorithm was developed for continuous case only. Therefore we have to make some special arrangements for the two discrete parameters. Pole pair number is an integer between 20 and 80. For this parameter we can use an approximation of the continuous case and always round the value to the nearest integer. However, the number of slots per pole per phase can only be 1, 2 or 3. Therefore we have to run the optimization three times with each value to find the optimal parameters.

The implementation began with defining the required variables for the swarm and the limitations. The easy part was to fill the solution space with random particles. After implementing the basic algorithm some problems emerged.

The first disturbing observation was that the computer resources weren't used to their full potential as the calculations were performed mostly on one core at a time. After some research the Matlab Parallel Computing Toolbox, and especially the parfor loop, was found. This allowed for the parallelization of the generator simulation and somewhat improved the performance. Not everything is parallelizable though as the particles have to be synchronized. 

The next problem was how to enforce the modeling limitations. The initial attempt was to enforce the parameters to stay inside the defined problem space but this is against the nature of the algorithm. As a result the particles would often hit the hard limits and gather near the edges. The decision was made to check the boundaries but skip the calculations and assign a bad fitness value if any of the parameters were out of bounds.

After the change in handling the boundaries it was noticed that the swarm diverges for some reason. In spite of penalizing out of bounds values the particles seemed to wander far from the desired area. This had to be looked into.

After finding the document by Engelbrecht ~\cite{pitfalls} it became evident that the particle velocities exploded. Using a simpler model $f(x,y)=-x^2-y^2$ yielded similar results, so the next step was to introduce velocity clamping. By changing the fraction of the maximum allowed velocity related to the size of the parameter space it is possible to adjust the particle convergence speed and the tendency of the particles wandering off.


Running the provided simulation model was inconsistent performance wise. Usually the simulation was calculated in much less than one second but sometimes the calculation could take several seconds, though. This might happen because of nearly singular matrices in the model but the issue is out of our reach.


%------------------------------------------------

\section{Time and Teamwork Management}

As mentioned before we had some challenges allocating time for the project. The deadlines for the different phases have been delayed. Here we summarize the time usage up to date.

\textbf{Janne} Project planning took 8 hours. If the first lectures are counted in, the total time for the first phase is 14 hours. Familiarization with the PSO algorithm took 1 hour and there have been 14 hours of programming. Writing the intermediate report took 5 hours. Total time allocated for the course is 34 hours.

\textbf{Eero}







%------------------------------------------------

\section{Risk management}

Here we discuss the risks that have realized. As mentioned in the project plan the System Dynamics project work required quite a lot effort and prevented this project from advancing. Then there have been many other things distracting us from this particular project.

There have been no illnesses that would have hindered the project.

At this time, we have opted out of the multiobjective optimization as it would require a much more advanced approach to the optimization.

There has already been some need to rewrite program code as things haven't always worked out the way they were intended to.

The diverging of the PSO algorithm caused some headache until velocity clamping was introduced to the system. 

Git hasn't caused any major problems yet.


%----------------------------------------------------------------------------------------
%	REFERENCE LIST
%----------------------------------------------------------------------------------------

\begin{thebibliography}{99} % Bibliography - this is intentionally simple in this template

\bibitem{pso}
 Kennedy, J. and Eberhart, R.
  \emph{Particle Swarm Optimization},
 Proceedings of IEEE international conference on neural networks. Vol. 4. No. 2. 1995. 
 
\bibitem{pitfalls}
 Engelbrecht, A. \emph{Particle Swamr Optimization: Pitfalls and Convergence Aspects.}
 Department of Computer Science, University of Pretoria, South Africa
 
 
\end{thebibliography}

%----------------------------------------------------------------------------------------

\end{multicols}

\end{document}